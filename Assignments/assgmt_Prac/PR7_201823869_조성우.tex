% Options for packages loaded elsewhere
\PassOptionsToPackage{unicode}{hyperref}
\PassOptionsToPackage{hyphens}{url}
%
\documentclass[
]{article}
\usepackage{lmodern}
\usepackage{amssymb,amsmath}
\usepackage{ifxetex,ifluatex}
\ifnum 0\ifxetex 1\fi\ifluatex 1\fi=0 % if pdftex
  \usepackage[T1]{fontenc}
  \usepackage[utf8]{inputenc}
  \usepackage{textcomp} % provide euro and other symbols
\else % if luatex or xetex
  \usepackage{unicode-math}
  \defaultfontfeatures{Scale=MatchLowercase}
  \defaultfontfeatures[\rmfamily]{Ligatures=TeX,Scale=1}
\fi
% Use upquote if available, for straight quotes in verbatim environments
\IfFileExists{upquote.sty}{\usepackage{upquote}}{}
\IfFileExists{microtype.sty}{% use microtype if available
  \usepackage[]{microtype}
  \UseMicrotypeSet[protrusion]{basicmath} % disable protrusion for tt fonts
}{}
\makeatletter
\@ifundefined{KOMAClassName}{% if non-KOMA class
  \IfFileExists{parskip.sty}{%
    \usepackage{parskip}
  }{% else
    \setlength{\parindent}{0pt}
    \setlength{\parskip}{6pt plus 2pt minus 1pt}}
}{% if KOMA class
  \KOMAoptions{parskip=half}}
\makeatother
\usepackage{xcolor}
\IfFileExists{xurl.sty}{\usepackage{xurl}}{} % add URL line breaks if available
\IfFileExists{bookmark.sty}{\usepackage{bookmark}}{\usepackage{hyperref}}
\hypersetup{
  pdftitle={PR7 - 함수},
  pdfauthor={조성우},
  hidelinks,
  pdfcreator={LaTeX via pandoc}}
\urlstyle{same} % disable monospaced font for URLs
\usepackage[margin=1in]{geometry}
\usepackage{color}
\usepackage{fancyvrb}
\newcommand{\VerbBar}{|}
\newcommand{\VERB}{\Verb[commandchars=\\\{\}]}
\DefineVerbatimEnvironment{Highlighting}{Verbatim}{commandchars=\\\{\}}
% Add ',fontsize=\small' for more characters per line
\usepackage{framed}
\definecolor{shadecolor}{RGB}{248,248,248}
\newenvironment{Shaded}{\begin{snugshade}}{\end{snugshade}}
\newcommand{\AlertTok}[1]{\textcolor[rgb]{0.94,0.16,0.16}{#1}}
\newcommand{\AnnotationTok}[1]{\textcolor[rgb]{0.56,0.35,0.01}{\textbf{\textit{#1}}}}
\newcommand{\AttributeTok}[1]{\textcolor[rgb]{0.77,0.63,0.00}{#1}}
\newcommand{\BaseNTok}[1]{\textcolor[rgb]{0.00,0.00,0.81}{#1}}
\newcommand{\BuiltInTok}[1]{#1}
\newcommand{\CharTok}[1]{\textcolor[rgb]{0.31,0.60,0.02}{#1}}
\newcommand{\CommentTok}[1]{\textcolor[rgb]{0.56,0.35,0.01}{\textit{#1}}}
\newcommand{\CommentVarTok}[1]{\textcolor[rgb]{0.56,0.35,0.01}{\textbf{\textit{#1}}}}
\newcommand{\ConstantTok}[1]{\textcolor[rgb]{0.00,0.00,0.00}{#1}}
\newcommand{\ControlFlowTok}[1]{\textcolor[rgb]{0.13,0.29,0.53}{\textbf{#1}}}
\newcommand{\DataTypeTok}[1]{\textcolor[rgb]{0.13,0.29,0.53}{#1}}
\newcommand{\DecValTok}[1]{\textcolor[rgb]{0.00,0.00,0.81}{#1}}
\newcommand{\DocumentationTok}[1]{\textcolor[rgb]{0.56,0.35,0.01}{\textbf{\textit{#1}}}}
\newcommand{\ErrorTok}[1]{\textcolor[rgb]{0.64,0.00,0.00}{\textbf{#1}}}
\newcommand{\ExtensionTok}[1]{#1}
\newcommand{\FloatTok}[1]{\textcolor[rgb]{0.00,0.00,0.81}{#1}}
\newcommand{\FunctionTok}[1]{\textcolor[rgb]{0.00,0.00,0.00}{#1}}
\newcommand{\ImportTok}[1]{#1}
\newcommand{\InformationTok}[1]{\textcolor[rgb]{0.56,0.35,0.01}{\textbf{\textit{#1}}}}
\newcommand{\KeywordTok}[1]{\textcolor[rgb]{0.13,0.29,0.53}{\textbf{#1}}}
\newcommand{\NormalTok}[1]{#1}
\newcommand{\OperatorTok}[1]{\textcolor[rgb]{0.81,0.36,0.00}{\textbf{#1}}}
\newcommand{\OtherTok}[1]{\textcolor[rgb]{0.56,0.35,0.01}{#1}}
\newcommand{\PreprocessorTok}[1]{\textcolor[rgb]{0.56,0.35,0.01}{\textit{#1}}}
\newcommand{\RegionMarkerTok}[1]{#1}
\newcommand{\SpecialCharTok}[1]{\textcolor[rgb]{0.00,0.00,0.00}{#1}}
\newcommand{\SpecialStringTok}[1]{\textcolor[rgb]{0.31,0.60,0.02}{#1}}
\newcommand{\StringTok}[1]{\textcolor[rgb]{0.31,0.60,0.02}{#1}}
\newcommand{\VariableTok}[1]{\textcolor[rgb]{0.00,0.00,0.00}{#1}}
\newcommand{\VerbatimStringTok}[1]{\textcolor[rgb]{0.31,0.60,0.02}{#1}}
\newcommand{\WarningTok}[1]{\textcolor[rgb]{0.56,0.35,0.01}{\textbf{\textit{#1}}}}
\usepackage{graphicx,grffile}
\makeatletter
\def\maxwidth{\ifdim\Gin@nat@width>\linewidth\linewidth\else\Gin@nat@width\fi}
\def\maxheight{\ifdim\Gin@nat@height>\textheight\textheight\else\Gin@nat@height\fi}
\makeatother
% Scale images if necessary, so that they will not overflow the page
% margins by default, and it is still possible to overwrite the defaults
% using explicit options in \includegraphics[width, height, ...]{}
\setkeys{Gin}{width=\maxwidth,height=\maxheight,keepaspectratio}
% Set default figure placement to htbp
\makeatletter
\def\fps@figure{htbp}
\makeatother
\setlength{\emergencystretch}{3em} % prevent overfull lines
\providecommand{\tightlist}{%
  \setlength{\itemsep}{0pt}\setlength{\parskip}{0pt}}
\setcounter{secnumdepth}{-\maxdimen} % remove section numbering

\title{PR7 - 함수}
\author{조성우}
\date{2020년 4월 30일}

\begin{document}
\maketitle

\hypertarget{uxd568uxc218uxc640-uxc0acuxc6a9uxc790uxc815uxc758-uxd568uxc218}{%
\section{함수와 사용자정의
함수}\label{uxd568uxc218uxc640-uxc0acuxc6a9uxc790uxc815uxc758-uxd568uxc218}}

\hypertarget{uxd568uxc218}{%
\subsection{함수}\label{uxd568uxc218}}

\begin{itemize}
\tightlist
\item
  특정 목적에 맞게 생성된 연산과정의 집합
\item
  ex) mean 함수 : 모든 원소의 합을 원소의 개수로 나눔
\end{itemize}

\#\#사용자 정의 함수 * 사용자의 편의에 따라 직접 작성하여 사용하는 함수
* 함수명 \textless- function(인수)\{연산과정\} 형태로 작성(한가지만
연산할 경우\{\}로 묶지 않아도 됨) * 연산과정으로 나오는 결과값을
return,print,cat등으로 반환하는 형태가 이상적

\hypertarget{uxc5d0uxc2dc1.-uxb450-uxc22buxc790uxb97c-uxbe44uxad50uxd574-uxb354uxd070uxc218uxb97c-uxbc18uxd658uxd558uxb294-uxd568uxc218}{%
\subsubsection{에시1. 두 숫자를 비교해 더큰수를 반환하는
함수}\label{uxc5d0uxc2dc1.-uxb450-uxc22buxc790uxb97c-uxbe44uxad50uxd574-uxb354uxd070uxc218uxb97c-uxbc18uxd658uxd558uxb294-uxd568uxc218}}

\begin{Shaded}
\begin{Highlighting}[]
\CommentTok{#2개의 숫자를 인수로 받아서 더 큰수를 반환하는 함수}
\NormalTok{compare <-}\StringTok{ }\ControlFlowTok{function}\NormalTok{(x,y) }\ControlFlowTok{if}\NormalTok{(x}\OperatorTok{>}\NormalTok{y) }\KeywordTok{cat}\NormalTok{(x) }\ControlFlowTok{else} \KeywordTok{cat}\NormalTok{(y)}
\KeywordTok{compare}\NormalTok{(}\DecValTok{10}\NormalTok{,}\DecValTok{20}\NormalTok{)}
\end{Highlighting}
\end{Shaded}

\begin{verbatim}
## 20
\end{verbatim}

\hypertarget{uxc608uxc2dc2.-uxd3c9uxade0uxac12uxacfc-uxd45cuxc900uxc624uxcc28uxb97c-uxacc4uxc0b0uxd558uxb294-uxd568uxc218}{%
\subsubsection{예시2. 평균값과 표준오차를 계산하는
함수}\label{uxc608uxc2dc2.-uxd3c9uxade0uxac12uxacfc-uxd45cuxc900uxc624uxcc28uxb97c-uxacc4uxc0b0uxd558uxb294-uxd568uxc218}}

\begin{Shaded}
\begin{Highlighting}[]
\CommentTok{#표준오차 = 표준편차 / 표본의크기}

\NormalTok{se <-}\StringTok{ }\ControlFlowTok{function}\NormalTok{(x)\{}
\NormalTok{  tmp.sd <-}\KeywordTok{sd}\NormalTok{(x) }\CommentTok{#표준편차}
\NormalTok{  tmp.N <-}\StringTok{ }\KeywordTok{length}\NormalTok{(x) }\CommentTok{#표본크기}
\NormalTok{  tmp.se <-}\StringTok{ }\NormalTok{tmp.sd }\OperatorTok{/}\StringTok{ }\KeywordTok{sqrt}\NormalTok{(tmp.N) }\CommentTok{#평균의 표준오차}
\NormalTok{\}}

\NormalTok{A <-}\StringTok{ }\KeywordTok{c}\NormalTok{(}\DecValTok{1}\NormalTok{,}\DecValTok{2}\NormalTok{,}\DecValTok{3}\NormalTok{,}\DecValTok{4}\NormalTok{,}\DecValTok{5}\NormalTok{,}\DecValTok{6}\NormalTok{,}\DecValTok{7}\NormalTok{,}\DecValTok{8}\NormalTok{,}\DecValTok{9}\NormalTok{,}\DecValTok{10}\NormalTok{)}
\KeywordTok{cat}\NormalTok{(}\KeywordTok{se}\NormalTok{(A))}
\end{Highlighting}
\end{Shaded}

\begin{verbatim}
## 0.9574271
\end{verbatim}

\hypertarget{uxc608uxc2dc3.-uxb370uxc774uxd130-uxd504uxb808uxc784uxc758-uxc55euxb4a4-3uxac1cuxc758-uxb370uxc774uxd130uxb97c-uxb9acuxc2a4uxd2b8uxb85c-uxbcf4uxc5ecuxc8fcuxb294-uxd568uxc218}{%
\subsubsection{예시3. 데이터 프레임의 앞뒤 3개의 데이터를 리스트로
보여주는
함수}\label{uxc608uxc2dc3.-uxb370uxc774uxd130-uxd504uxb808uxc784uxc758-uxc55euxb4a4-3uxac1cuxc758-uxb370uxc774uxd130uxb97c-uxb9acuxc2a4uxd2b8uxb85c-uxbcf4uxc5ecuxc8fcuxb294-uxd568uxc218}}

\begin{Shaded}
\begin{Highlighting}[]
\NormalTok{head_tail <-}\StringTok{ }\ControlFlowTok{function}\NormalTok{(x)\{}
\NormalTok{  front <-}\StringTok{ }\KeywordTok{head}\NormalTok{(x,}\DecValTok{3}\NormalTok{)}
\NormalTok{  rear <-}\StringTok{ }\KeywordTok{tail}\NormalTok{(x,}\DecValTok{3}\NormalTok{)}
\NormalTok{  F_R <-}\StringTok{ }\KeywordTok{list}\NormalTok{(front,rear) }\CommentTok{#2개의 데이터프레임 릿트로 묶음}
  \KeywordTok{print}\NormalTok{(F_R) }\CommentTok{#묶은 리스트 반환}
\NormalTok{\}}

\KeywordTok{head_tail}\NormalTok{(mtcars)}
\end{Highlighting}
\end{Shaded}

\begin{verbatim}
## [[1]]
##                mpg cyl disp  hp drat    wt  qsec vs am gear carb
## Mazda RX4     21.0   6  160 110 3.90 2.620 16.46  0  1    4    4
## Mazda RX4 Wag 21.0   6  160 110 3.90 2.875 17.02  0  1    4    4
## Datsun 710    22.8   4  108  93 3.85 2.320 18.61  1  1    4    1
## 
## [[2]]
##                mpg cyl disp  hp drat   wt qsec vs am gear carb
## Ferrari Dino  19.7   6  145 175 3.62 2.77 15.5  0  1    5    6
## Maserati Bora 15.0   8  301 335 3.54 3.57 14.6  0  1    5    8
## Volvo 142E    21.4   4  121 109 4.11 2.78 18.6  1  1    4    2
\end{verbatim}

\hypertarget{uxc608uxc2dc4.-uxd640uxc218-uxd310uxbcc4-uxd568uxc218}{%
\subsubsection{예시4. 홀수 판별
함수}\label{uxc608uxc2dc4.-uxd640uxc218-uxd310uxbcc4-uxd568uxc218}}

\begin{Shaded}
\begin{Highlighting}[]
\NormalTok{oddnum <-}\StringTok{ }\ControlFlowTok{function}\NormalTok{(x)\{}
  \ControlFlowTok{if}\NormalTok{(x}\OperatorTok\DecValTok{2}\OperatorTok{==}\DecValTok{1}\NormalTok{)\{ }\CommentTok{#2로 나눈 나머지가 1이면}
    \KeywordTok{return}\NormalTok{(T)}
\NormalTok{  \}}\ControlFlowTok{else}\NormalTok{ \{ }\CommentTok{#그렇지 않으면}
    \KeywordTok{return}\NormalTok{(f)\} }\CommentTok{#F를 반환}
\NormalTok{\}}
\end{Highlighting}
\end{Shaded}

\hypertarget{scope-of-variable}{%
\subsection{scope of variable}\label{scope-of-variable}}

\begin{itemize}
\tightlist
\item
  함수 바깥에서 생성된 변수는 같은함수 안에서는 언제나 사용가능
\item
  함수 안에서 생성된 변수는 함수가 종료되면 사라짐(local variable은
  휘발성)
\item
  함수 내에서 생성된 변수가 사라지지 않게 하려면
  ``\textless\textless-''을 할당 연산자로 사용(global variable로 할당)
\end{itemize}

\begin{Shaded}
\begin{Highlighting}[]
\CommentTok{# <- 할당 연산자 사용}
\NormalTok{scopetest <-}\ControlFlowTok{function}\NormalTok{(x)\{}
\NormalTok{  a <-}\StringTok{ }\DecValTok{10}
  \KeywordTok{print}\NormalTok{(a)}
  \KeywordTok{print}\NormalTok{(x)}
\NormalTok{\}}

\KeywordTok{scopetest}\NormalTok{(}\DecValTok{9}\NormalTok{)}
\end{Highlighting}
\end{Shaded}

\begin{verbatim}
## [1] 10
## [1] 9
\end{verbatim}

\begin{Shaded}
\begin{Highlighting}[]
\CommentTok{#print(a) #주석 제거후 함수실행하여 메세지 확인할것}

\CommentTok{# <<- 할당 연산자 사용}
\NormalTok{scopetest <-}\StringTok{ }\ControlFlowTok{function}\NormalTok{(x)\{}
\NormalTok{  a <<-}\StringTok{ }\DecValTok{10}
  \KeywordTok{print}\NormalTok{(a)}
  \KeywordTok{print}\NormalTok{(x)}
\NormalTok{\}}

\KeywordTok{scopetest}\NormalTok{(}\DecValTok{9}\NormalTok{)}
\end{Highlighting}
\end{Shaded}

\begin{verbatim}
## [1] 10
## [1] 9
\end{verbatim}

\begin{Shaded}
\begin{Highlighting}[]
\KeywordTok{print}\NormalTok{(a)}
\end{Highlighting}
\end{Shaded}

\begin{verbatim}
## [1] 10
\end{verbatim}

\hypertarget{uxd568uxc218uxc758-default-uxac12-uxc124uxc815}{%
\subsubsection{함수의 default 값
설정}\label{uxd568uxc218uxc758-default-uxac12-uxc124uxc815}}

\begin{itemize}
\tightlist
\item
  인수를 입력하지 않았을때 자동으로 적용되는 값을 default라고 함
\item
  함수작성시 ``인수=T 또는 인수=10'' 이런식으로 미리 인수에 적용될 값을
  입력
\end{itemize}

\begin{Shaded}
\begin{Highlighting}[]
\NormalTok{add10 <-}\StringTok{ }\ControlFlowTok{function}\NormalTok{(}\DataTypeTok{x=}\DecValTok{10}\NormalTok{)x}\OperatorTok{+}\DecValTok{10}
\KeywordTok{add10}\NormalTok{()}
\end{Highlighting}
\end{Shaded}

\begin{verbatim}
## [1] 20
\end{verbatim}

\hypertarget{pr7-uxc5f0uxc2b5uxbb38uxc81c}{%
\section{PR7 연습문제}\label{pr7-uxc5f0uxc2b5uxbb38uxc81c}}

\hypertarget{uxbb38uxc81c-1}{%
\subsection{문제 1}\label{uxbb38uxc81c-1}}

\begin{itemize}
\tightlist
\item
  PR3의 연습문제 1번을 활용한 문제입니다.
\item
  벡터 prices에 저장된 값은 2020-03-01 부터 2020-03-06 까지 bitcoin의
  종가이다. *prices \textless- c(11905000.0, 1973000.0, 12190000.0,
  12700000.0, 12303000.0, 12604000.0)
\item
  힌트를 참고하여 순서대로 2020-03-02 부터 2020-03-06 까지 5일간의
  수익률을 구하는 in\_rate 함수를 작성하세요.
\item
  함수를 사용하여 증가율을 출력해 주세요. \emph{HINT 수익율 = ((금일의
  종가 - 전일의 종가)/전일의 종가)}100
\end{itemize}

\begin{Shaded}
\begin{Highlighting}[]
\NormalTok{prices <-}\StringTok{ }\KeywordTok{c}\NormalTok{(}\FloatTok{11905000.0}\NormalTok{, }\FloatTok{11973000.0}\NormalTok{, }\FloatTok{12190000.0}\NormalTok{, }\FloatTok{12700000.0}\NormalTok{, }\FloatTok{12303000.0}\NormalTok{, }\FloatTok{12604000.0}\NormalTok{)}

\NormalTok{in.rate <-}\StringTok{ }\ControlFlowTok{function}\NormalTok{(a)\{             }\CommentTok{#Define}
\NormalTok{  price_today <-}\StringTok{ }\NormalTok{a[}\OperatorTok{-}\DecValTok{1}\NormalTok{]}
\NormalTok{  price_yesterday <-}\StringTok{ }\NormalTok{a[}\OperatorTok{-}\KeywordTok{length}\NormalTok{(a)]}
\NormalTok{  returns <-}\StringTok{ }\NormalTok{(price_today }\OperatorTok{-}\StringTok{ }\NormalTok{price_yesterday) }\OperatorTok{/}\StringTok{ }\NormalTok{price_yesterday }\OperatorTok{*}\StringTok{ }\DecValTok{100}
  \KeywordTok{return}\NormalTok{(returns)}
\NormalTok{\}}

\KeywordTok{in.rate}\NormalTok{(prices)}
\end{Highlighting}
\end{Shaded}

\begin{verbatim}
## [1]  0.5711886  1.8124113  4.1837572 -3.1259843  2.4465578
\end{verbatim}

\hypertarget{uxbb38uxc81c-2}{%
\subsection{문제 2}\label{uxbb38uxc81c-2}}

\begin{itemize}
\tightlist
\item
  PR5의 연습문제 1번을 활용하는 문제입니다.
\item
  URL을 입력받아서 해당 웹사이트의 Table list를 반환하는 함수
  read.html.tables를 작성하세요.
\item
  웹사이트의 URL을 입력하여 결과값을 저장하고, 특정 테이블을 head를
  사용해서 출력해주세요.(웹사이트 자율)
\end{itemize}

\begin{Shaded}
\begin{Highlighting}[]
  \KeywordTok{library}\NormalTok{(XML)}
  \KeywordTok{library}\NormalTok{(httr)}
\end{Highlighting}
\end{Shaded}

\begin{verbatim}
## Warning: package 'httr' was built under R version 3.6.3
\end{verbatim}

\begin{Shaded}
\begin{Highlighting}[]
\NormalTok{read.html.tables <-}\StringTok{ }\ControlFlowTok{function}\NormalTok{(URL)\{      }\CommentTok{#웹의 content를 불러오는 read.html.tables 함수를 Define해줍니다.}

\NormalTok{  html_source <-}\StringTok{ }\KeywordTok{GET}\NormalTok{(URL)}
\NormalTok{  tabs <-}\StringTok{ }\KeywordTok{readHTMLTable}\NormalTok{(}\KeywordTok{rawToChar}\NormalTok{(html_source}\OperatorTok{$}\NormalTok{content), }\DataTypeTok{stringsAsFactors =}\NormalTok{ F)}
  \KeywordTok{return}\NormalTok{(tabs)}
\NormalTok{\}}

\NormalTok{KoreaPopulation <-}\KeywordTok{read.html.tables}\NormalTok{(}\StringTok{"https://www.worldometers.info/world-population/south-korea-population/"}\NormalTok{) }\CommentTok{#read.html.tables 함수를 call하여 변수에 할당}
\NormalTok{ForecastKorPOP <-}\StringTok{ }\NormalTok{KoreaPopulation[}\DecValTok{2}\NormalTok{]       }\CommentTok{#테이블 리스트중 한국 인구예측 테이블을 서브세팅해줍니다.}

\KeywordTok{names}\NormalTok{(ForecastKorPOP) <-}\StringTok{ }\NormalTok{ForecastKorPOP    }\CommentTok{#해당 테이블의 이름이 NULL이므로 이름을 지어줍니다.}
\KeywordTok{head}\NormalTok{(ForecastKorPOP,}\DecValTok{5}\NormalTok{) }
\end{Highlighting}
\end{Shaded}

\begin{verbatim}
## $...
##    Year Population Yearly %  Change Yearly Change Migrants (net) Median Age
## 1  2020 51,269,185           0.09 %        43,877         11,731       43.7
## 2  2019 51,225,308           0.10 %        53,602         11,731       41.4
## 3  2018 51,171,706           0.15 %        75,291         11,731       41.4
## 4  2017 51,096,415           0.22 %       112,958         11,731       41.4
## 5  2016 50,983,457           0.32 %       160,364         11,731       41.4
## 6  2015 50,823,093           0.51 %       255,491         80,237       40.8
## 7  2010 49,545,636           0.34 %       168,913        -31,309       38.0
## 8  2005 48,701,073           0.55 %       264,366         16,245       34.8
## 9  2000 47,379,241           0.90 %       417,344         31,886       31.9
## 10 1995 45,292,522           1.08 %       474,821         14,284       29.3
## 11 1990 42,918,419           1.02 %       422,803         34,116       27.0
## 12 1985 40,804,402           1.41 %       551,759         18,578       24.3
## 13 1980 38,045,607           1.46 %       533,389        -33,027       22.1
## 14 1975 35,378,661           1.90 %       636,596        -41,988       19.9
## 15 1970 32,195,681           2.19 %       660,025        -16,369       19.0
## 16 1965 28,895,558           2.67 %       713,209        -13,827       18.4
## 17 1960 25,329,515           3.32 %       762,989         62,079       18.6
## 18 1955 21,514,570           2.29 %       460,637         86,590       18.9
##    Fertility Rate Density (P/Km짼) Urban Pop % Urban Population
## 1            1.11              527      81.8 %       41,934,110
## 2            1.21              527      81.6 %       41,805,375
## 3            1.21              526      81.4 %       41,678,226
## 4            1.21              526      81.3 %       41,552,264
## 5            1.21              524      81.3 %       41,426,777
## 6            1.23              523      81.3 %       41,301,851
## 7            1.17              510      81.9 %       40,601,614
## 8            1.21              501      81.4 %       39,622,010
## 9            1.50              487      79.6 %       37,729,427
## 10           1.68              466      78.2 %       35,441,319
## 11           1.57              441      73.9 %       31,696,103
## 12           2.23              420      64.9 %       26,474,831
## 13           2.92              391      56.7 %       21,582,191
## 14           4.00              364      48.0 %       16,997,155
## 15           4.65              331      40.7 %       13,110,502
## 16           5.60              297      32.4 %        9,351,713
## 17           6.33              261      27.7 %        7,022,058
## 18           5.65              221      24.4 %        5,251,885
##    Country's Share of World Pop World Population South KoreaGlobal Rank
## 1                        0.66 %    7,794,798,739                     28
## 2                        0.66 %    7,713,468,100                     28
## 3                        0.67 %    7,631,091,040                     28
## 4                        0.68 %    7,547,858,925                     27
## 5                        0.68 %    7,464,022,049                     27
## 6                        0.69 %    7,379,797,139                     27
## 7                        0.71 %    6,956,823,603                     26
## 8                        0.74 %    6,541,907,027                     25
## 9                        0.77 %    6,143,493,823                     24
## 10                       0.79 %    5,744,212,979                     24
## 11                       0.81 %    5,327,231,061                     24
## 12                       0.84 %    4,870,921,740                     23
## 13                       0.85 %    4,458,003,514                     23
## 14                       0.87 %    4,079,480,606                     23
## 15                       0.87 %    3,700,437,046                     24
## 16                       0.87 %    3,339,583,597                     24
## 17                       0.83 %    3,034,949,748                     24
## 18                       0.78 %    2,773,019,936                     24
\end{verbatim}

\hypertarget{uxbb38uxc81c-3}{%
\subsection{문제 3}\label{uxbb38uxc81c-3}}

\begin{itemize}
\item
  어떤 문제를 수치 계산으로 풀지 않고 확률(난수)를 이용해서 푸는 것을
  몬테카를로법이라 한다.이방법으로 원주율 파이를 구할 수 있다.
  \textbf{각 도형의 면적} * 반지름이ㅣ 1인 원 면적의 1/4 = 1/4 x pi X
  1\^{}2 = pi/4 * 한 변이 1인 정사각형의 면적 = 1\^{}2 = 1 \textbf{공식
  유도} 1. 1/4원 내부에 표시된 난수 개수 : a 2. 1/4원 외부에 표시된 난수
  개수 : b 3. pi/4:1 = a:a+b 4. pi = 4a/a+b = 4a/n \textbf{이미지는
  PR본문참고}

  \begin{itemize}
  \item
    점을 뽑을 갯수 n을 입력받아 파이를 추정하는 함수 mc\_pi를
    작성하세요.
  \item
    점을 100개,1000개,10000개,100000개 뽑았을 떄 추정된 파이를 출력하고
    실제 파이값과 가까워지는지 확인하세요. * HINT1 : 난수생성함수
    runif(n,min=0,max=1) * 0 에서 1 사이의 난수를 n개 생성 (default) *
    HINT2 : 좌표평면상의 점이므로,각각 n개의 x값과 y값이 필요합니다.
    두점과 원점 (0,0)사이의 거리는 다음과 같이 구할 수 있습니다.

\begin{verbatim}
 *distance = sqrt(x^2 + y^2)
\end{verbatim}
  \end{itemize}
\end{itemize}

\begin{Shaded}
\begin{Highlighting}[]
\NormalTok{vec <-}\StringTok{ }\KeywordTok{c}\NormalTok{(}\DecValTok{100}\NormalTok{, }\DecValTok{1000}\NormalTok{, }\DecValTok{10000}\NormalTok{, }\DecValTok{100000}\NormalTok{)}

\ControlFlowTok{for}\NormalTok{(v }\ControlFlowTok{in}\NormalTok{ vec)\{}
  \KeywordTok{cat}\NormalTok{(}\StringTok{"n = "}\NormalTok{,v,}\StringTok{"일 때 추정된 PI :"}\NormalTok{,}\StringTok{"mc_pi(v)"}\NormalTok{,}\StringTok{"}\CharTok{\textbackslash{}n}\StringTok{"}\NormalTok{)}
  \CommentTok{#print(paste0("n =",v,"일 때 추정된 PI :",mc_pi(v),"\textbackslash{}n")))}
\NormalTok{\}}
\end{Highlighting}
\end{Shaded}

\begin{verbatim}
## n =  100 일 때 추정된 PI : mc_pi(v) 
## n =  1000 일 때 추정된 PI : mc_pi(v) 
## n =  10000 일 때 추정된 PI : mc_pi(v) 
## n =  1e+05 일 때 추정된 PI : mc_pi(v)
\end{verbatim}

\end{document}
